% Context
The rapid transmission of \acrlong{cov} has caused different governments to screen the infected population from healthy ones. While \acrfull{pcr} tests are used to identify the infected, it requires a long, difficult and painful process to get the results. Other tests like \acrfull{rdt} for \acrlong{cov} are not considered reliable. 
% What is already known about the question
\acrfull{cxr} images can be used to diagose the \acrlong{covid} as the damages done by corona virus is present prominently in lungs.
% The main Reasons for the study 
Due to high transmission of \acrshort{covid}, the laboratory images have been highly available in the laboratory repository.
% Central Questions
As \acrfull{ai} techniques are being used in the medicine and bioinformatics, classification of \acrshort{cxr} image for \acrshort{covid} damages using computational methods could result in fast result processing from the existent methods.
% Research Methods
The research will explore different deep learning networks using \acrshort{cnn} for CXR images and develop a state-or-art model for properly classifying CXR image in context of Nepal.
% Findings
% Significance
The model shall be published in research repository of Gandaki Province which can be used by radio specialists for \acrshort{covid} tests.